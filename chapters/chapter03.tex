% chktex-file 21

In this chapter we will present some mathematical necessities that will allow us
to discuss our problem at a higher level of rigor and detail. The tools we will
need are some partial differential equation theory, elements of functional
analysis, and ideas from the calculus of variations.

\section{Function Spaces}\label{sec:funspaces}
Since this thesis is concerned with whether a certain model contains particular
solutions, we will need to specify the space in which we are looking for
solutions. Are we looking for continuous functions, H\"{o}lder continuous
functions, or perhaps even functions that are discontinuous? In this section we
will address these questions and motivate the space that is of particular
interest.

In this chapter we will \textit{mostly} work with functions of one variable
because, in our Q-Ball work, that is all we need, but all definitions and
theorems can be extended to functions of more variables. Most of the time the
necessary background, and notation bogs down, and makes harder to see the
meaning the definitions and theorems are trying to get at.

We begin with a few definitions that are of utmost importance in order to grasp
further concepts. In the work to come, pertaining to Q-Balls we will need a
notion of ``size'' that will measure the strength of the solutions. We will here
on out assume the underlying structure of a vector space\footnote{If it is not
clear such a space is a vector space, we will provide an argument.}.
\begin{definition}\label{def:normspace}
    A vector space \(V\) over the field \(\mathbb{F}\) equipped with a function
    \(\|\cdot\|: V\to \mathbb{F}\) is said to be a \textbf{normed vector space}
    if \(\|\cdot \|\) satisfies the following properties. 
    \begin{itemize}
        \item \(\|\alpha x\| = |\alpha|\|x\|\)\hfill \normalfont{(Absolutely Homogenuous)}
        \item \(\|x\| \geq 0\) and equals 0 if and only if \(x = 0\) \hfill \normalfont{(Positive Definite)}
        \item \(\|x + y\| \leq \|x\| + \|y\|\) \hfill \normalfont{(Subadditive)\footnote{Also known as, satisfying the Triangle Inequality.}}
    \end{itemize}
    Where \(x,y\in V\) and \(\alpha\in\mathbb{F}\).
\end{definition}
With this definition, we see the fundamental ideas or notions of ``length''
embodied. We can now provide a definition which allows one to have a notion of
``direction'' in the space of question.
\begin{definition}\label{def:innerprodspace}
    A vector space \(V\) over the field \(\mathbb{F}\) equipped with a function
    \(\langle\cdot, \cdot\rangle:V\times V\to \mathbb{F}\) is said to be an
    \textbf{inner product space} if \(\langle\cdot, \cdot\rangle\) satisfies the
    following properties.
    \begin{itemize}
        \item \(\langle x,y\rangle = \overline{\langle y,x\rangle}\) \hfill \normalfont{(Conjugate Symmetric)}
        \item \(\langle\alpha x + \beta y, z\rangle = \alpha \langle x,z\rangle + \beta\langle y,z\rangle\) \hfill \normalfont{(Linear in first entry)}
        \item \(\langle x,x\rangle\geq 0\) and equals 0 if and only if \(x = 0\) \hfill \normalfont{(Positive Definite)}
    \end{itemize}
    Where \(x,y,z\in V\) and \(\alpha, \beta\in\mathbb{F}\).
\end{definition}
It is important to note here that once an inner product is defined on a vector
space, we can immediately define a norm by \(\|x\|_{\langle\cdot,\cdot\rangle} = \sqrt{\langle x, x\rangle}\).

The next definition provided allows us to ``fill in the gaps''. To give meaning
to this statement look at \(\mathbb{R}\). If we build \(\mathbb{R}\) starting
with the natural numbers, adding in the integers, and again adding in the
quotients, then we are left with a set that resembles \(\mathbb{R}\), but has
lots of missing pieces. To fill those in, we take all Cauchy Sequences
i.e.\ sequences such that for any \(\varepsilon >0\) there exists an
\(N\in\mathbb{N}\) such that for \(n,m\geq N\) we have
\(|a_n - a_m| < \varepsilon\), in \(\mathbb{R}\) and add the limits to our
existing set. This ``fills in'' the gaps and provides a much nicer space to work
with. This motivates the following definition.
\begin{definition}\label{def:complete}
    A normed vector space \((V, \|\cdot\|)\) is \textbf{complete} if all Cauchy
    sequences converge with respect to the metric
    \(d_{\|\cdot\|}(x,y)\coloneqq\|x - y\|\).
\end{definition}
Forcing the spaces we work with to be complete allows one to not worry about
pedagogical counterexamples that make basic operations and constructions much
more tedious and cumbersome. We can now add this condition to
Definitions~\ref{def:normspace} and~\ref{def:innerprodspace} to obtain further
restricted spaces.
\begin{definition}\label{def:banach}
    A \textbf{Banach Space} is a complete, normed vector space.
\end{definition}
\begin{definition}
    A \textbf{Hilbert Space} is a complete, inner product space.
\end{definition}
It is important to remember that we can define completeness on an innerproduct
space, because the inner product gives rise to the norm, which gives rise to the
metric \(d_{\|\cdot\|}\) defined in Definition~\ref{def:complete}.

\subsection{\texorpdfstring{\(L^p\)}{Lp} spaces}\label{subsec:lp}
To begin our work with function spaces we introduce one of the most fundamental
spaces, the space of Lebesgue integrable functions. In particular, given a
measure space \((S,\Sigma, \mu)\)\footnote{Here \(S\) is the base set,
\(\Sigma\) is a collections of subsets of \(S\) that form a \(\sigma\)-algebra
(a collections of subsets that contain the base set, is closed under
complements, and closed under countable unions) and \(\mu\) is a function
\(\mu:\Sigma\to[0,\infty]\) such that \(\mu(E)\geq 0\),
\(\mu(\varnothing) = 0\), \(\mu\left(\bigcup_{i\in\mathbb{N}}E_i\right) = \sum_{i\in\mathbb{N}}\mu(E_i)\),
and finally we have the conditions that \(\mu(S) = 1\).}, for each
\(p\in[1,\infty)\) we define the Lebesgue norm of a function % chktex 9
\(f:S\to\mathbb{R}\) as
\begin{equation}
    \|f\|_{L^p(S)}^p\coloneqq \int_S|f|^p\dd{\mu}.
\end{equation}
If this value is finite, we write \(f\in L^p(S)\). Ideally these functions would
form a space with some structure, so that we can add them, and multiply them by
scalars, i.e., a vector space.
\begin{lemma}
    The set \(L^p(\Omega)\) (\(p\in[1,\infty)\)) with addition and scalar % chktex 9
    multiplication defined by
    \begin{align}
        (f + g)(x)    & \coloneqq f(x) + g(x) \\
        (\alpha f)(x) & \coloneqq \alpha\cdot f(x).
    \end{align}
    forms a vector space.
\end{lemma}
\begin{proof}
    We begin by showing \(\alpha f\in L^p(\Omega)\).
    \begin{align*}
        \|\alpha f\|_{L^p(\Omega)} & = \left(\int_\Omega |\alpha f|^p\dd{\mu}\right)^\frac{1}{p} \\
                                   & \leq |\alpha|\left(\int_\Omega |f|^p\dd{\mu}\right)^\frac{1}{p}\\
                                   & = |\alpha|\cdot\|f\|_{L^p(\Omega)} < \infty
    \end{align*}
    Since it's Lebesgue integral is finite, it is also in the space. We know
    show vector addition also yields a function within \(L^p(\Omega)\).
    \begin{align*}
        \|f + g\|_{L^p(\Omega)} & = \left(\int_\Omega |f + g|^p\dd{\mu}\right)^\frac{1}{p} \\
                                & \leq \left(\int_\Omega |2\max{(|f|,|g|)}|^p\dd{\mu}\right)^\frac{1}{p} \\
                                & = 2\left(\int_\Omega |f|^p\dd{\mu}\right)^\frac{1}{p} \\
                                & = 2\|f\|_{L^p(\Omega)} < \infty
    \end{align*}
    Where we have taken \(\max{(|f|,|g|)} = |f|\) arbitrarily. The rest of the
    vector space axioms are satisfied almost trivially, and hence for
    \(p\in[1,\infty)\), we have that \(L^p\) is a vector space. % chktex 9
\end{proof}

\begin{remark}
    For \(p\in(0,1)\) the space \(L^p\) does \textit{not} form a vector space
    because of the fact that the norm fails to be subadditive.
\end{remark}

An extremely important, and I would argue the \textit{best}\footnote{Not
subjective whatsoever.} value of \(p\) is 2. This is because \(L^2\) is a
Hilbert space\footnote{Remember this is a complete inner product space, pretty
much the best version of a vector space you can imagine.}. The functions that
live here are often called square integrable functions and have applications
across all areas of mathematics and are extremely prevalent in physics, in
particular quantum mechanics.

These functions are extremely important, yet most of the time these functions
are not even differentiable. As we will see in Section~\ref{sec:consmin} the
solutions we wish to work with need to be ``differentiable'' once, in some sense
of the word. Thus, the space \(L^p\) is much too large, and we need to throw
away some.

\subsection{Sobolev Spaces}\label{subsec:weakds}
In this section we introduce the function spaces we will be working in for the
following thesis. In order to introduce this space we will first introduce a
weaker notion of differentiability that, again, expands our previous notions.
Before this, we provide some more fundamental notions.
\begin{definition}
    Given a function a topological space \(X\) and a continuous function
    \(f:X\to\mathbb{R}\) we define the \textbf{support} of \(f\) to be the
    closure of the set of points in \(X\) where the function does not vanish, i.e.
    \begin{equation}
        \supp{f}\coloneqq \cl{\{x\in X | f(x) \neq 0\}} = \cl{f^{-1}(\mathbb{R}\setminus\{0\})}.
    \end{equation}
\end{definition}
A very important object is then a function which ``lives'' on a ``small'' or
``confined'' subset of the domain. The following definition makes this precise.
\begin{definition}
    A function \(f:X\to\mathbb{R}\) is said to have \textbf{compact support} if
    \(\supp{f}\) is a compact subset of \(X\).
\end{definition}
To illustrate this concept we provide the following example.
\begin{example}
    Let \(f:\mathbb{R}\to\mathbb{R}\) be defined by
    \begin{equation}
        f(x) = \begin{cases}
            g(x) & x\in[a,b] \\
            0    & \text{otherwise}
        \end{cases}
    \end{equation}
    where \(g\in\mathrm{C}^0([a,b])\). This is trivially a function of compact
    support as \(\supp{f} = [a,b]\subset\mathbb{R}\) is a compact set.
\end{example}

\begin{example}
    Let \(f:\mathbb{R}\to\mathbb{R}\) be defined by \(x\mapsto x^2\). It is easy
    to to see \(\supp{f} = \mathbb{R}\setminus \{0\}\). This set is not only
    open, but also unbounded and hence not compact in \(\mathbb{R}\). Hence this
    map does not have compact support.
\end{example}

An important characterization of functions with compact support is their
behavior on or near the boundary of the compact subset on which the function
lives. We provide the following lemma without proof.

\begin{lemma}\label{lem:comsupp}
    Suppose \(f:\mathbb{R}\to\mathbb{R}\) is a continuous function compactly
    supported on \(U\subset\mathbb{R}\). Then \(f\) vanishes on the boundary of
    \(U\) i.e., \(f|_{\partial U} = 0\). 
\end{lemma}

Before the definition we introduce two new spaces. Some of the ``nicest''
functions are ones that we can differentiate as many times as we'd like. To
denote infinitely differentiable functions, we have \(\mathrm{C}^\infty(U)\) for
some open subset \(U\subset \mathbb{R}\). A subset of these functions are the
ones that vanish on the boundary of \(U\), i.e., by Lemma~\ref{lem:comsupp} they
have compact support and this is denoted \(\mathrm{C}_c^\infty(U)\).

In Subsection~\ref{subsec:lp} we saw \(L^p\) spaces. We now introduce a slightly
larger set of functions called \(L^p_\text{loc}(U)\) to mean functions who are
locally summable on \(U\). This means functions which are Lebesgue integrable on
all compact subsets of \(U\). To fully understand this space we provide an
example.

\begin{example}
    The space \(L^1(\mathbb{R})\) is defined by functions that satisfy
    \begin{equation*}
        \int_\mathbb{R}|f(x)|\dd{x} < \infty.
    \end{equation*}
    In order for a function \(\phi\) to satisfy this property, it must decay to
    0 as \(|x|\to\infty\). The space \(L^1_\text{loc}(\mathbb{R})\) is much
    larger. Note compact subsets of \(\mathbb{R}\) are all of the form \([a,b]\)
    for some \(a,b\in\mathbb{R}\). Thus in order for \(\phi\) to be in
    \(L^1_\text{loc}(\mathbb{R})\) we must have
    \begin{equation*}
        \int_a^b|\phi(x)|\dd{x} < \infty
    \end{equation*}
    for all \(a,b\in\mathbb{R}\). This condition is much less restrictive as we
    now allow for objects such as constant functions, periodic functions, and
    any continuous function defined on \(\mathbb{R}\).
\end{example}

With these new spaces we can now define our weakened notion of derivative.
\begin{definition}
    If \(f,g \in L^1_\mathrm{loc}(U)\), we call \(g\) the
    \textbf{weak derivative} of \(f\) provided
    \begin{equation}
        \int_U f\varphi'\dd{x} = -\int_U g\varphi\dd{x}
    \end{equation}
    for all \(\varphi\in\mathrm{C}_c^\infty(U)\).
\end{definition}
This definition comes from integration by parts \(\int u\dd{v} = uv|_\partial - \int v\dd{u}\)
where the function we are integrating against, in our case \(\varphi'\) vanishes
on the boundary. Thus we can throw away the boundary term and we are left with
another integral. This sense of derivative is much weaker than ordinary
derivatives and allows us to weakly derive things like step functions or ramp
functions to say the least. We can now easily prove the uniqueness of these
objects.
\begin{lemma}
    The weak derivative of \(f\), should it exist, is unique.
\end{lemma}
\begin{proof}
    Let us assume \(g_1, g_2\) are both weak derivatives of \(f\). Then we have
    \begin{equation}
        \int_U f\varphi'\dd{x} = -\int_U g_1\varphi\dd{x} = -\int_U g_2\varphi\dd{x}
    \end{equation}
    for all \(\varphi\in \mathrm{C}_c^\infty(U)\) and hence, by linearity of the
    Lebesgue integral,
    \begin{equation}
        \int_U\left(g_1 - g_2\right)\varphi\dd{x} = 0
    \end{equation}
    and thus \(g_1\overset{\text{a.e.}}{=}g_2\)\footnote{The ``a.e.'' over the
    ``\(=\)'' means these functions are equal \textbf{a}lmost
    \textbf{e}verywhere, or put differently, they are the same in the integral
    sense, but not necessarily pointwise.}.
\end{proof}

With the basic ideas of weak differentiability we can construct the Sobolev
Space. Remember \(L^p\) spaces were far too big and contained functions we
wished not to consider. Thus in defining Sobolev spaces we will first start in
\(L^p\) and require some orders of weak derivatives also lie in various \(L^p\)
spaces.
\begin{definition}
    The \textbf{Sobolev Space} \(W^{k,p}(U)\) (\(k,p\in\mathbb{N}\)) consists of
    functions \(f\in L^p_\mathrm{loc}(U)\) such that \(f^{(i)}\) exists in the
    weak sense and also belongs to \(L^p(U)\).
\end{definition}
This space admits a natural norm that makes it a Banach space (\ref{def:banach}).
\begin{equation}
    \|f\|_{W^{k,p}(U)}^p \coloneqq \sum_{i = 0}^k\int_U |f^{(i)}|^p\dd{x}
\end{equation}
An immediate important subset of the Sobolev space is \(W^{k,p}_0(U)\) which
denotes the closure of \(\mathrm{C}^\infty_c(U)\) in \(W^{k,p}(U)\). Another
particularly interesting case is when \(p = 2\), and in this case we write
\(H^k(U) = W^{k,2}(U)\) where we use the letter \(H\) because this space
inherits a completeness, and an inner product from \(L^2\) and hence it is a
Hilbert space.

\section{The Calculus of Variations}\label{sec:cov}
In this section we present the underlying ideas that allow us to solve
(\ref{eq:DE}). This method provides an alternative way to view partial
differential equations that is often insightful and lends itself to Lagrangian
systems.

The basic idea of partial differential equations is to solve
\begin{equation}\label{eq:PDEprob}
    D[u] = 0
\end{equation}
where \(D\) is some differential operator. The calculus of variations treats
this differential operator as the ``derivative'' of some functional
\(I[\cdot]\), i.e., \(I'[u] = A[u]\). The fundamental problem of PDE's
(\ref{eq:PDEprob}) is then written as \(I'[u] = 0\). The advantage this method
has over attempting to directly solve PDE's is that often times finding critical
points may be easier than a direct proof of existence of solutions.

To better understand the basic idea of this we provide an illustrative example.
\begin{example}
    Suppose we wish to solve \(u'' = \sin{u}\) for some function
    \(u\in\mathrm{C}^2(\mathbb{R})\). Directly this problem can be very
    difficult, so let us try and find a functional, whose derivative represents
    the operator \(A = \partial_x^2 - \sin(\cdot)\). If we define
    \begin{equation}
        I[u] = \int \frac{1}{2}(u')^2 - \cos{u}\dd{x}
    \end{equation}
    then take its functional derivative as we learned in
    Chapter~\ref{chap:fields} we see the Euler-Lagrange equations of this
    functional yields the original partial differential equation. Hence if we
    are able to find minima of \(I[\cdot]\), then we have found solutions to the
    original problem.
\end{example}
In this example we found \(I\) to be the integral of another function. As this
is normally the case, we denote by \(L\), the ``\textit{Lagrangian}'' of such a
functional. Thus for the above example we have \(L = \frac{1}{2}(u')^2 - \cos{u}\).
This function is called the Lagrangian simply because resembles the action
equation we saw in Equation (\ref{eq:action}).

\subsection{When does \texorpdfstring{\(\inf = \min\)}{inf = min}?}
In this section we study when a functional attains its minima. If it does, then
\(\inf = \min\), so really you can stop reading this section now since I
answered the main question. In this section we will study further properties
which ensure a functional does attain its minima.

It is easy to see that given, even an infinitely differentiabl function, it need
not attain its minima as seen in \(f(x) = \e^x\). Thus, in order to force it to
attain its minima we must control the function at ``large'' input. A way to do
this is to say, as \(|x|\to \infty\) we also have \(f(x)\to +\infty\). The
following definition generalizes this to functionals. 

\begin{definition}\label{def:coercive}
    A functional \(I\) with Lagrangian \(L\) is said to be coercive if, for a
    given \(q\), there exist constants \(\alpha\) and \(\beta\) such that
    \begin{equation}
        L(f', f, x) \geq \alpha |f'|^q - \beta.
    \end{equation}
\end{definition}
This definition immediately leads to an alternative, but equivalent description
in terms of the functional.
\begin{lemma}
    A functional is coercive if and only if it satisfies a coercivity condition
    i.e.
    \begin{equation}
        I[f] \geq \delta \|f'\|^q_{L^q(U)} - \gamma
    \end{equation}
\end{lemma}
\begin{proof}
    To show these are equivalent, we simply integrate the Lagrangian and use the
    inequality.
    \begin{align*}
        I[f] & = \int_U L(f',f,x)\dd{x} \\
             & \geq \int_U \alpha |f'|^q - \beta \dd{x} \\
             & = \delta\|f'\|_{L^q(U)}^q - \gamma \qedhere
    \end{align*}
\end{proof}
