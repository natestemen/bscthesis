Throughout this thesis we have accomplished many things. We began with a
thorough introduction to field theory, assuming only a rudimentary understanding
of classical mechanics. Here we developed ideas that utilized in nearly all
areas of theoretical, and mathematical physics and are paramount for
understanding current research. The highlights were Gauge Theory, the field
theoretic Lagrangian, and of course Noether's Theorem, all of which play a very
important role in the work that followed.

To best understand the work that followed, a chapter of the necessary
mathematical prerequisites followed where we introduced fundamental ideas from
functional analysis and the calculus of variations in order to discuss our
problem at hand with a level of rigor.

With the fundamentals from both mathematics, and physics we were able to
introduce Q-Balls. These extremely imporant objects were developped
systematically using the tool from the previous chapters that allowed us to
calculate many important physics quantities such as total energy, angular
momentum, and the stabilizing Noether charge. Once the basic quantities were
calculated, the constrained minimization problem was introduced, and we proved
it was well defined.

The fact that the problem was well defined allowed us to move forward and to
look for solutions numerically. Using a finite element formalism we reduced the
problem to a multivariable calculus minimization problem which is easily solved
numerically. We were then able to compute Q-Ball profiles and their associated
parameters which allow us to understand the objects much better. With the
constrained minimization problem set up we were able to explore parameter space
not previously studied in the literature. In particular we studied how the
Q-Ball changes as one varies the ``strength'' or ``power'' of the object.

While the end of this thesis has come, the work on the problem is certainly not.
There is much more to explore in this field, and in particular this study of
Q-Balls. Further questions are, which potentials have solutions, and are they
qualitatively different from the ones found here? Are the solutions found
unique?\footnote{One way to explore this is to test a variety of points
satisfying the constraint and seeing if they all lead to the same solution.} Can
we decrease the numerical error? There are of course many more interesting
questions to be had, but unfortunately we only had a finite time to answer some.