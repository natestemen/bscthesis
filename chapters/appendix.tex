% chktex-file 21

In this chapter we provide a few ``random'' facts, theorems and arguments that
are not paramount to this thesis, but help it along.

\section{Some Theorems}\label{sec:apptheorems}
\begin{theorem}\label{th:actionexist}
    Let \(X\subseteq\mathbb{R}^n\) be open, and \(\alpha, \beta \in\mathbb{R}_+\)
    with \(\alpha < \beta\). If \(L:[\alpha,\beta]\times X\times \mathbb{R}^n\to\mathbb{R}\)
    is continuously differentiable, then the integral
    \begin{equation}
        S[x(t)]\coloneqq\int_\alpha^\beta L(t, x(t), \dot{x}(t))\dd{t}
    \end{equation}
    exists for all \(x\in\mathrm{C}^1([\alpha,\beta],X)\).
\end{theorem}
This theorem justifies calling \(\mathrm{C}^1(\mathbb{R})\) the domain of the
action as we did in Section~\ref{sec:stationary}. The next theorem is a slightly
more rigorous justification as to when the Euler-Lagrange
equations~\ref{eq:eulerlag} hold.
\begin{theorem}
    If \(x\in\mathrm{C}^1([\alpha,\beta],X)\) is a local extremum of \(S\) as
    defined above, and moreover the function
    \begin{equation}
        [\alpha,\beta]\ni t\mapsto \pdv{\dot{x}}L(t, x(t),\dot{x}(t))
    \end{equation}
    is continuously differentiable, then the following equation holds.
    \begin{equation}
        \pdv{x}L(t, x(t),\dot{x}(t)) = \dv{t}\pdv{\dot{x}}L(t, x(t),\dot{x}(t))
    \end{equation}
\end{theorem}

\begin{theorem}[Poincar\'{e} Inequality]
    Let \(p\in[1,\infty)\) and \(\Omega\) a bounded subset of \(\mathbb{R}\). % chktex 9
    Then there exists a constant \(C\), dependent on \(\Omega\) and \(p\), so
    that for every function \(u\in W^{1,2}(\Omega)\) we have the following
    inequality.
    \begin{equation}
        \|u\|_{L^p(\Omega)} \leq C\|u'\|_{L^p(\Omega)}
    \end{equation}
\end{theorem}


\section{Some Arguments}\label{sec:appargs}
In this section we recreate, with more detail, the original arguments for the
existence of Q-Balls as proposed in~\cite{coleman}.

As Q-Balls were proposed in~\cite{coleman} these objects are described by a
field \(\phi\) that is some constant value inside a volume \(B\), and 0 outside.
There exact energy is given by (\ref{eq:nospinenergy}), and in this
approximation reduces to
\begin{equation}\label{eq:Eapprox}
    E = \frac{1}{2}\omega^2\phi^2B + VB
\end{equation}
and similarly for the Noether Charge (\ref{eq:nospincharge}) we have the following.
\begin{equation}\label{eq:Qapprox}
    Q = \omega \phi^2 B
\end{equation}
The fundamental idea of a Q-Ball is that it's energy configuration is lower than
that of the separate particles, and hence we should minimize the energy for a
given value of the charge \(Q\). In order to do this we eliminate \(\omega\) by
inserting (\ref{eq:Qapprox}) into (\ref{eq:Eapprox}) to obtain
\begin{equation}
    E = \frac{1}{2}\frac{Q^2}{\phi^2 B} + VB.
\end{equation}
Now we minimize this with respect to the volume \(B\) to find it is minimized when
\begin{equation}
    B = \frac{Q}{\sqrt{2\phi^2 V}}.
\end{equation}
With this volume the energy then reads
\begin{align}
    E & = \frac{1}{2}\frac{Q^2}{\phi^2}\frac{\sqrt{2\phi^2 V}}{Q} + \frac{VQ}{\sqrt{2\phi^2 V}} \\
      & = Q\sqrt{2\phi^2 V}\left( \frac{1}{2\phi^2} + \frac{V}{2\phi^2 V}\right) \\
      & = Q \sqrt{\frac{2V}{\phi^2}}.
\end{align}
The last step is then to minimize this value with respect to \(\phi\) in order
to obtain
\begin{equation}
    E = \min_\phi Q \sqrt{\frac{2V}{\phi^2}}
\end{equation}
which is where the restriction on \(\omega\) comes from as seen in
(\ref{eq:omegabounds}). The upper bound for \(\omega\) arises from considering
the qualitative shape of the potential, and a more detailed description of its
derivation can be found in~\cite{coleman, qball6}.

This construction of Q-Balls allows the energy configuration of the ``Q-matter''
as Coleman calls it to be lower than that of the separate particles floating
around in space.