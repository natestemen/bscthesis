% chktex-file 21
% chktex-file 3

In this Chapter we use the constrained minimization problem put forth in
Section~\ref{sec:consmin} to numerically solve the equation of motion. This
numerical solution provides further evidence for the existence of Q-Vortex
solitons and the profiles allow us to better understand the nature of the
objects.

\section{Finite Element Formalism}
As we saw in Chapter~\ref{chap:qballs}, in particular Section~\ref{sec:consmin}
the constrained minimization approach to the problem is well defined because of
the coercivity of the functional \(I[\,\cdot\,]\) along with its lower
semi-continuity. This means we can utilize it as a tool to numerically solve the
equation of motion (\ref{eq:DE}) if we can reduce the infinite dimensional
domain \(W^{1,2}_0(0,R)\) to finite dimensional.

To attack this problem we will begin by taking a subspace
\(A\subset W^{1,2}_0(0,R)\) spanned by \(n\) vectors \(\{\psi_i\}_{i = 1}^n\).
We take the \(\psi\)'s to be orthonormal with respect to the following inner
product
\begin{equation}\label{eq:innerprod}
    \langle\alpha,\beta\rangle\coloneqq 2\pi\int_0^P r\alpha\beta\dd{r}
\end{equation}
where \(\alpha,\beta\in A\).
The form of (\ref{eq:innerprod}) is suggested by the Noether Charge we saw in
Chapter~\ref{chap:qballs} and \(\langle\varphi,\varphi\rangle = \|\varphi\|^2\)
corresponds to almost exactly the Noether Charge (\ref{eq:spincharge}). Using
this this basis we approximate functions \(f \in W_0^{1,2}(0, P)\) simply as a
linear combinations
\begin{equation}\label{eq:approx}
    f = \sum_{i = 1}^n a_i\psi_i
\end{equation}
with all \(a_i\in \mathbb{R}\). Inserting this approximation into our functional
(\ref{eq:actionfunc}) we can define a new multivariable function
\begin{equation}\label{eq:objfun}
    F(\mathbf{a}) \coloneqq I[f] = I\!\left[\sum_{i = 1}^n a_i\psi_i\right]
\end{equation}
where \(\mathbf{a} = (a_1, a_2,\ldots, a_n)^\intercal\in \mathbb{R}^n\) is
called the variational vector. Now our problem of minimizing the action over an
infinite dimensional function space becomes one of finite dimension. That is,
now we want to minimize a multi-variable function \(F:\mathbb{R}^n\to\mathbb{R}\)
which is easy (most of the time) numerically. To further restrict our study we
look for solutions with finite charge i.e.
\begin{equation}\label{eq:constraint}
    \langle f, f \rangle = \|f\|^2 = J_0 < \infty.
\end{equation}
Upon inserting (\ref{eq:approx}) into (\ref{eq:constraint}) we obtain the
following condition\footnote{using the fact that the inner product is linear in
    both entries, and the fact that the basis is orthonormal i.e.,
    \(\langle \psi_i,\psi_j\rangle = \delta_{ij}\).} on the variational vector
\(\mathbf{a}\).
\begin{equation}\label{eq:approxcons}
    \sum_{i = 1}^n a_i^2 = J_0 \iff \|\mathbf{a}\|^2_{\mathbb{R}^n} = J_0
\end{equation}
This condition realizes our problem now as a constrained minimization problem
where, given solitonic charge \(J_0\), we can find the solution if we minimize
over all functions satisfying (\ref{eq:approxcons}). Written out, our problem
(\ref{eq:consmin}) has been simplified to
\begin{equation}\label{eq:problem}
    \mathbf{a}_{\min, J_0}\coloneqq \min_{\mathbf{a}\in\mathbb{R}^n}\left\{F(\mathbf{a}) \,\mid\, \|\mathbf{a}\|^2_{\mathbb{R}^n} = J_0\right\}.
\end{equation}
We have used the notation \(\mathbf{a}_{\min, J_0}\) to mean the vector
\(\mathbf{a}\) that minimizes \(F\) for a prescribed \(J_0\).

Now we note two very important facts. First, \(F\) is a continuous function.
This can be seen by expanding (\ref{eq:objfun}) and writing it as a sum over the
basis. Since the basis is known ahead of time this expansion yields a polynomial
in the \(a_i\)'s and since it is a polynomial, it is clearly continuous. Second,
the constraint (\ref{eq:approxcons}) defines a compact set, in particular a
sphere of radius \(\sqrt{J_0}\) centered at the origin. With a continuous
function defined over a compact set, we know the minima must be attained and
hence this problem is well defined and has a solution.


\section{Implementation}\label{sec:imp}
The code written to implement the constrained minimization problem
(\ref{eq:problem}) was written in Python with use of NumPy and
SciPy\footnote{Both of which are pronounced with a ``pi'' sound, not ``pee''.
    Num-``pi'', not numpee please.}. We now explain, and ellucidate the process we
used to obtain the numerical solutions.

We begin with NumPy arrays initialized to
\(\psi_i = \sin{\frac{i\pi x}{r_\text{max}}}\) with \(i\) running in the set
\(\{1,2,\ldots, n\}\) before we orthonormalize them using the  the Gram-Schmidt
Procedure, and the inner product given in (\ref{eq:innerprod}). All integration
is handled using Simpson's Rule, and the \texttt{scipy.integrate.simps} built in
function. In the plot below we see the result of Gram-Schmidt orthonormalization
procedure under this inner product for a basis of size 10.
\begin{figure}[H]
    \centering
    \includegraphics[width=0.95\linewidth]{basis.pdf}
    \caption{10 Orthonormal Basis Functions with \(r_\text{max} = 10\)}\label{fig:basis}
\end{figure}

With the basis at hand we can build the objective function (\ref{eq:objfun}),
that is the function we wish to minimize. To satisfy (\ref{eq:approxcons}) we
begin with the initial guess given by \(\left(\sqrt{\frac{J_0}{n}}, \sqrt{\frac{J_0}{n}}, \ldots, \sqrt{\frac{J_0}{n}}\right)\).
With an objective function, an initial guess, and a constraint, we can use
SciPy's \texttt{scipy.optimize.minimize} to carry out the
minimization\footnote{In order to have this function succeed for large values of
    \(n\), that is a large basis set, the maximum number of iterations must be
    increased to reach the minimum.}. To reproduce the results in~\cite{spinningq},
in particular, Figure 6, we chose parameters \(a = 2, b = 1.1, \lambda = 1\) and
with \(J_0 = 30\) we obtain the following solution.
\begin{figure}[H]
    \centering
    \includegraphics[width=0.95\linewidth]{qballprof30.pdf}
    \caption{\(N = 1\) Excitation of Q-Vortex Solution}\label{fig:profile30}
\end{figure}
This plot was generated with 50 basis functions and an \(x\)-axis sampled 1000
times, i.e. \(\dd{x} = 0.01\). With this solution we can then go on to calculate
the Lagrange multiplier \(\omega^2\) that we saw in Lemma~\ref{lem:nonlineig}.
To calculate \(\omega^2\), we take the \(\delta I\) and \(\delta J\) as defined
in Lemma~\ref{lem:nonlineig}, and take the test function \(\psi\) to be exactly
\(\phi\). This may seem like cheating, however \(\delta I = \chi\delta J\)
should hold for all \(\psi\), and in particular \(\psi = \phi\). We can then
numerically calculate \(\delta I\) and \(\delta J\), calculate the Lagrange
multiplier by taking the quotient \(\chi = \frac{\delta I}{\delta J}\), and then
using the relation \(4\pi\chi = \omega^2\) we have \(\omega = \sqrt{4\pi\chi}\).
For the plot above this yields \(\omega = 0.7216\).

We can then run this procedure for multiple values of \(N\) to further
understand how the Q-Vortex behaves with larger \(N\) values. We note how, as we
increase \(N\), the peak of the soliton moves outward similar to how objects
with more angular momentum tend to increase their radii.
\begin{figure}[H]
    \centering
    \includegraphics[width=0.95\linewidth]{qballprofs.pdf}
    \caption{First Four Fundamental Q-Vortex Solutions}\label{fig:profiles}
\end{figure}
Now that we have solutions at hand, it is important to understand the error
associated with them. To measure the error we use the following idea and metric.
If the solutions computed above are accurate, they should of course satisfy the
equation of motion (\ref{eq:DE}). Thus we can add everything to one side,
multiply by \(r^2\), square the differential equation, and integrate\footnote{We
    multiply by \(r^2\) to make our lives easier so we don't have to deal with the
    inverse powers of \(r\) numerically.}.
\begin{equation}
    \mathrm{error} = \int_0^R \left(r\left(r\phi_{r}\right)_r + \omega^2r^2\phi - N^2\phi - \lambda r^2\left(6\phi^5 - 4a\phi^3 + 2b\phi\right)\right)^2\dd{r}
\end{equation}
We can then tabulate the associated values for each solution shown in
Figure~\ref{fig:profiles}.
\begin{table}[H]
    \centering
    \begin{tabular}{c c c}            \toprule
        \(N\) & \(\omega\) & error   \\ \midrule
        1     & 0.72156    & 0.00626 \\ \midrule
        2     & 0.85271    & 0.00084 \\ \midrule
        3     & 1.01243    & 0.00021 \\ \midrule
        4     & 1.16983    & 0.01622 \\ \bottomrule
    \end{tabular}
    \caption{Parameters for \(J_0 = 30\)}\label{tab:params}
\end{table}

It is then natural to ask, for a given \(N\), how does omega change as we vary
the power \(J_0\). Below we see that as the power is increased the soliton peak
flattens and moves outward.
\begin{figure}[H]
    \centering
    \includegraphics[width=0.95\linewidth]{varypower.pdf}
    \caption{Flat Top Emergence of Q-Vortex Soliton}\label{fig:flattops}
\end{figure}
The parameters for this plot is shown in the table below.
\begin{table}[H]
    \centering
    \begin{tabular}{c c c}              \toprule
        \(J_0\) & \(\omega\) & error   \\ \midrule
        10      & 1.06898    & 0.00002 \\ \midrule
        20      & 0.80058    & 0.00498 \\ \midrule
        30      & 0.72156    & 0.00626 \\ \midrule
        40      & 0.68074    & 0.00137 \\ \midrule
        50      & 0.65481    & 0.00447 \\ \midrule
        60      & 0.63783    & 0.03409 \\ \midrule
        70      & 0.62205    & 0.02264 \\ \midrule
        80      & 0.61019    & 0.00427 \\ \midrule
        90      & 0.60444    & 0.06596 \\ \bottomrule
    \end{tabular}
    \caption{Parameters for \(N = 1\)}\label{tab:paramsN}
\end{table}
It is easy to see in this table that as the power increases, omega not only
decreases, but it looks as though it tends to some number. To investigate this
further we performed a finer scan to understand how the power \(J_0\) effects
\(\omega\).
\begin{figure}[H]
    \centering
    \includegraphics[width=0.95\linewidth]{omegavJ.pdf}
    \caption{\(\omega\) vs. \(J_0\) Parameter Space Scan}\label{fig:paramspace}
\end{figure}
As this plot shows, as we increase the \(J_0\), \(\omega\) falls very quickly at
first, and then more slowly as it dips slightly below 0.6. In particular the
minimum this curve achieves (not including small deviations) is roughly 0.57.





